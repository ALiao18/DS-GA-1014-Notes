% !TEX root = ../main.tex
\section{Vector Spaces}

\subsection{Vector Spaces}

\begin{definition}[Vector]
    A \textbf{vector} $\bar{v}$ is a mathematical object defined by the operations of vector addition and scalar multiplication. Vectors have a geometric interpretation as arrows in space and a numerical interpretation as ordered lists of real numbers.
\end{definition}

\begin{definition}[Vector Space]
    A \textbf{vector space} $V$ is a set of vectors satisfying the following axioms for all $x,y,z \in V$ and scalars $a,b,c,d \in \mathbb{R}$:
    \begin{multicols}{2}
    \begin{enumerate}
        \item Closure under addition: $x+y \in V$
        \item Commutativity: $x+y = y+x$
        \item Associativity: $(x+y)+z = x+(y+z)$
        \item Additive identity: $\exists\,0 \in V$ such that $x+0=x$
        \item Additive inverse: $\exists\,(-x)\in V$ such that $x+(-x)=0$
        \item Closure under scalar multiplication: $cx \in V$
        \item Compatibility: $c(dx) = (cd)x$
        \item Multiplicative identity: $1 \cdot x = x$
        \item Distributivity: $a(x+y)=ax+ay$
        \item Scalar distributivity: $(a+b)x=ax+bx$
    \end{enumerate}
    \end{multicols}
    Both vectors and vector spaces are defined through operations rather than representation.
\end{definition}

\begin{definition}[Subspace]
    A \textbf{subspace} $S\subseteq V$ is a subset that:
    \begin{enumerate}
        \item contains the zero vector,
        \item is closed under vector addition,
        \item is closed under scalar multiplication.
    \end{enumerate}
    If these conditions hold, $S$ is a vector space.
\end{definition}

\subsubsection*{Examples}
\begin{itemize}
    \item $\mathbb{R}^n$ is a subspace of itself.
    \item $\{0\}$ is a subspace of any vector space.
    \item Any line through the origin is a subspace of $\mathbb{R}^2$.
\end{itemize}

\subsection{Span and Linear Dependence}

\begin{definition}[Linear Combination]
    A vector $y\in V$ is a \textbf{linear combination} of $x_1,\dots,x_k$ if
    \[
    y=\sum_{i=1}^{k}\alpha_i x_i,
    \]
    for scalars $\alpha_i\in\mathbb{R}$.
\end{definition}

\begin{definition}[Span]
    The \textbf{span} of vectors $x_1,\dots,x_n$ is the set of all linear combinations.
    \[
    \operatorname{span}(x_1,\dots,x_n)
    = \{ \alpha_1x_1 + \dots + \alpha_nx_n : \alpha_i\in\mathbb{R}\},
    \]
    the smallest subspace of $V$ that contains them.
\end{definition}

\begin{definition}[Linear Dependence]
    Vectors $x_1,\dots,x_k$ are \textbf{linearly dependent} if not all scalars
    $\alpha_1,\dots,\alpha_k$ are zero and
    \[
    \alpha_1x_1+\dots+\alpha_kx_k = 0.
    \]
    They are \textbf{linearly independent} otherwise.
\end{definition}

\subsubsection*{Examples}
\begin{enumerate}
    \item The set $(x)$ is linearly independent iff $x\neq 0$.
    \item The set $(x,-x)$ is always linearly dependent because $x + (-x)=0$.
\end{enumerate}

\subsection{Basis and Dimension}

\begin{definition}[Basis]
    A family $x_1,\dots,x_n$ is a \textbf{basis} of $V$ if
    \begin{enumerate}
        \item $x_1,\dots,x_n$ are linearly independent,
        \item $\operatorname{span}(x_1,\dots,x_n)=V$.
    \end{enumerate}
\end{definition}

\begin{definition}[Dimension]
    If every basis of $V$ has $n$ vectors, we say $\dim(V)=n$.
    If no finite basis exists, $\dim(V)=+\infty$.
\end{definition}

\subsubsection{Properties}
Let $x_1,\dots,x_n \in V \text{ s.t. } \dim(V)=n$.
\begin{enumerate}
    \item If $x_1,\dots,x_n$ are linearly independent, then $x_1,\dots,x_n$ is a basis for $V$ (We get span for free).
    \item If $Span(x_1,\dots,x_n)=V$, then $(x_1,\dots,x_n)$ is the basis of $V$ (we get linear independence for free).
\end{enumerate}
To show that a family of vectors $x_1,\dots,x_k$ form a basis, we need to show:
\begin{enumerate}
    \item $k=n$
    \item linear independence
    \item $Span(x_1,\dots,x_k)=V$
\end{enumerate}
From above we see that showing any 2 properties implies the third one. We can trivially show $k=n$, so we usually choose to show the easier of 2 and 3.

\begin{definition}[Line and Hyperplane]
    Let $S$ be a subspace.
    \begin{enumerate}
        \item If $\dim(S)=1$, $S$ is a \textbf{line}.
        \item If $\dim(S)=n-1$, $S$ is a \textbf{hyperplane}.
    \end{enumerate}
\end{definition}

\subsection{Coordinates of a vector in a basis}

\begin{theorem}[Uniqueness of Coordinates]
    If $v_1,\dots,v_n$ is a basis of $V$, then for every $x \in V$, there exists a unique vector $(\alpha_1,\dots,\alpha_n) \in \mathbb{R}^n$
    such that $x = \alpha_1v_1 + \dots + \alpha_n v_n$.
    We refer to $(\alpha_1,\dots,\alpha_n)$ as the \textbf{coordinates} of $x$.
\end{theorem}

\begin{proof}
    $Span(v_1,\dots,v_n) = V$ since it is the basis of $V$.
    Since $x$ is some linear combination of $v_1,\dots,v_n$, $x \in Span(v_1,\dots,v_n)$.
    Therefore, there exists some vector $\alpha$ such that the linear combination of $v$ and $\alpha$ is $x$.
    Suppose there is another set of vectors $\beta_1,\dots,\beta_n$ such that $x$ is a linear combination of $v$ and $\beta$.
    \[ x - x = (\alpha_1 v_1 + \dots + \alpha_n v_n) - (\beta_1 v_1 + \dots + \beta_n v_n)=0 \]
    Since $v$ is linearly independent, $\alpha_i = \beta_i$ for $1 \leq i \leq n$.
    Therefore there is a unique vector in each basis to obtain $x$.
\end{proof}